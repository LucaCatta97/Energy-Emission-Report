\newpage
\chapter{Conclusions}
In order to assess which of the previously developed cases is the most convenient, we compared their relative economic costs and the eventual $CO_2$ emissions. Moreover, we also considered a realistic situation where only diesel trucks are employed (corresponding to the most common solution preferred nowadays), so to understand if there are any possible improvement margins and where to act, with the aim of making green solutions comparably convenient to the traditional ones.

The purchase cost of the HD trucks has been considered for each case study. Hydrogen trucks, provided by Nikola, are costing $268.782$ €, while Scania diesel Truck (Euro VI) costs around $153.000$ €.

\begin{table}
\centering
\begin{tabular}{|
>{\columncolor{bluepoli!40}}c |c|c|c|}
\hline
 &
  \cellcolor{bluepoli!40}\textbf{Unitary cost {[}€/MJ{]}} &
  \cellcolor{bluepoli!40}\textbf{Unitary cost {[}€/$kg_{H_2}${]}} &
  \cellcolor{bluepoli!40}\textbf{Total cost {[}€{]}} \\ \hline
\textbf{Crodo}       & 0,14 & 17,72 & 2.348.244,67 \\ \hline
\textbf{T. S. Adda}  & 0,08 & 10,99 & 1.455.707,54 \\ \hline
\textbf{PV - Puglia} & 0,15 & 19,48 & 2.581.128,75 \\ \hline
\textbf{PV - Milano} & 0,18 & 22,62 & 2.997.126,96 \\ \hline
\textbf{Diesel}      & 0,03 & -     & 1.019.416,67 \\ \hline
\end{tabular}
\caption{Resume of total and unitary costs}
\label{tab:cost_resume}
\end{table}

\begin{table}[h]
\centering
\begin{tabular}{|
>{\columncolor{bluepoli!40}}c|c|c|}
\hline
 &
  \cellcolor{bluepoli!40}\textbf{$CO_2$ emissions {[}kg{]}} &
  \cellcolor{bluepoli!40}\textbf{$CO_2$ reduction {[}\%{]}} \\ \hline
\multicolumn{1}{|c|}{\cellcolor{bluepoli!40}\textbf{Crodo}} &
  \multicolumn{1}{c|}{2.390.043,74} &
  \multicolumn{1}{c|}{99,13} \\ \hline
\multicolumn{1}{|c|}{\cellcolor{bluepoli!40}\textbf{T. S. Adda}}  & \multicolumn{1}{c|}{5.134.843,74}   & \multicolumn{1}{c|}{98,13} \\ \hline
\multicolumn{1}{|c|}{\cellcolor{bluepoli!40}\textbf{PV - Puglia}} & \multicolumn{1}{c|}{53.729,46}      & \multicolumn{1}{c|}{99,98} \\ \hline
\multicolumn{1}{|c|}{\cellcolor{bluepoli!40}\textbf{PV - Milano}} &
  \multicolumn{1}{c|}{0,00} &
  \multicolumn{1}{c|}{100,00} \\ \hline
\multicolumn{1}{|c|}{\cellcolor{bluepoli!40}\textbf{Diesel}}      & \multicolumn{1}{c|}{273.969.000,00} & \multicolumn{1}{c|}{0,00}  \\ \hline
\end{tabular}
\caption{\textit{Resume of $CO_2$ emissions}}
\label{tab:CO2_resume}
\end{table}

In Table are provided all the resumed information. It can be immediately noticed that, among green hydrogen production solution, hydroelectric cases appear to be cheaper with respect to the PV plant ones. This result is given by two factors: hydroelectric plants are already built, while in the case of photovoltaic we need to install new panels in order to feed our logistic node, but also because solar energy provision is not constant during the year, so a storage of energy (in form of electricity or hydrogen) is required. In fact, winter production can only partially respond to the demand of trucks, that we assumed to be constant along the year. Since a storage of electricity represented an excessive cost, we turned to gaseous hydrogen storage, that is more convenient, but requires also additional expenses for compression. This issues does not appear in the electricity production from dams, because the dams themselves are potential energy storage that can be spilled when needed.

What makes the TSA solution even more convenient that the Crodo one, is the external production of hydrogen from other private parties that we just buy. Assuming a shipment of the gas using traditional HD trucks, we show how emissions significantly increase. It must be specified that even though electricity production from hydroelectric plants is fully sustainable, the consumption done by our fictious company would force actual consumers to buy their power supply from the national grid (ecological worst case scenario), so our first two solutions cannot be said to be fully eco-friendly, but a significant reduction of $CO_2$ emissions would be achieved.

Analysing PV plant solution, only the Milano plant would grant zero emissions, because it is the one of the two not requiring a hydrogen transportation, which is a very impacting aspect also when calculating the annual cost.

It can be reasonably assumed that a local production and usage of hydrogen in Puglia, not requiring shipment costs, would be even more convenient, but it could not be coupled with alternative green energy sources as it can be done in Milano.

In the Crodo solution we can see that the higher reducible cost is the electrolyser’s but it appears that even a reduction of its cost of 100\% would not make this case as convenient as the diesel one; the only other relevant feature is the cost of electricity bought from the grid, but it cannot easily be forecasted.

The most relevant cost in TSA case is the transportation one. Since the annual cost of this case is comparable with the one of the diesel case, a substantial reduction of transport cost (around 30\%) would be enough to fill the gap, but such price changing is not to be expected in the coming years.

Finally, the main cost of the PV plants, as it can be easily guessed is the panel’s one, but only a reduction of 100\% (not a realistic case) of the cost would make these solutions comparable with the traditional one.

In conclusion, we assessed that conventional technologies nowadays are still more convenient than most of hydrogen production strategies (probably all, extending the reasoning also to unconsidered case studies) and the gap can be filled only by reducing economical costs of many different features, since we assessed that acting on just a single element is not enough to reach the goal. More research must be done and in order to foster societal use and private investments a public intervention is required.